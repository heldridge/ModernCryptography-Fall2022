\documentclass[11pt]{article}
\usepackage{url,amsmath,setspace,amssymb,fullpage,color,varwidth,mdframed,mathtools}
\usepackage{enumitem}
\usepackage{xcolor}
\usepackage{geometry}
\usepackage[
	n,
	operators,
	advantage,
	sets,
	adversary,
	landau,
	probability,
	notions,	
	logic,
	ff,
	mm,
	primitives,
	events,
	complexity,
	asymptotics,
	keys]{cryptocode}

\newgeometry{bottom=1.2cm, top=2cm}

\newcommand{\heading}[5]{
   \renewcommand{\thepage}{#1-\arabic{page}}
   \noindent
   \begin{center}
   \framebox[\textwidth]{
     \begin{minipage}{0.9\textwidth} \onehalfspacing
       {\bf CS 601.442/642 -- Modern Cryptography} \hfill #2

       {\centering \Large #5

       }\medskip

       {\it #3 \hfill #4}
     \end{minipage}
   }
   \end{center}
}

\newcommand{\scribe}[4]{\heading{#1}{#2}{}{ #4}{ #3}}

\newcommand{\proof}{{\bf Proof. }} %% To begin a proof write \proof
\newcommand{\qed}{\mbox{}\hspace*{\fill}\nolinebreak\mbox{$\rule{0.6em}{0.6em}$}} %%to end your proof write $\qed$.
\newcommand{\ma}{{\mathcal A}}
\newtheorem{lemma}{Lemma}
\newtheorem{theorem}[lemma]{Theorem}
\newtheorem{definition}{Definition}
\newtheorem{proposition}{Proposition}
\newtheorem{remark}{Remark}
\newtheorem{assumption}{Assumption}
\bibliographystyle{plain}
% \newcommand{\sample}{\xleftarrow{\$}}
\newcommand{\Gen}{\mathsf{Gen}}
\newcommand{\Enc}{\mathsf{Enc}}
\newcommand{\Dec}{\mathsf{Dec}}
\newcommand{\alice}{\mathsf{Alice}}
\newcommand{\bob}{\mathsf{Bob}}
\newcommand{\computeAkey}{\mathsf{ComputeAliceKey}}
\newcommand{\computeBkey}{\mathsf{ComputeBobKey}}
\newcommand{\msg}{\mathsf{msg}}
\newcommand{\out}{\mathsf{out}}
\newcommand{\secParam}{\lambda}
\newcommand{\ot}{\mathsf{OT}}

\newcommand{\hmenote}[1]{{\color{orange} Harry: #1}}

\begin{document}


\scribe{1}{Instructor:
Abhishek Jain}{Final Exam}{Deadline: December 21; 2022, 11:59 AM EST}

\paragraph{Instructions}
\begin{itemize}
    \item All submissions must be made via Gradescope. No late submissions will be accepted.
   \item Do not make public posts on Campuswire for the duration of this exam. You can still ask for clarifications via private posts on Campuswire.
    \item Please add the following declaration on the first page of your submission: \begin{center}
        \textit{``I have neither given nor received any unauthorized aid on this exam. I understand that this exam must be taken
          without the aid of any other online resources \textbf{besides} the lecture slides/videos, resources posted on the course website and the handouts sent via email. The work contained herein is wholly my own.
        I understand that violation of these rules, including using an unauthorized aid or collaborating with another person/student,
         may result in my receiving a 0 on this exam.''}
    \end{center}
    
    
\end{itemize}


\begin{enumerate}


\item {\bf (10 points) Key-Exchange:} 
Consider the following key-exchange protocol:
\begin{enumerate}
    \item Alice chooses $k,r\sample\{0,1\}^n$ at random, and sends $s=k\oplus r$ to Bob.
    \item Bob chooses $t\sample\{0,1\}^n$ at random and sends $u=s\oplus t$ to Alice.
    \item Alice computes $w =u \oplus r$ and sends $w$ to Bob.
    \item Alice outputs $k$ and Bob computes $w \oplus t$.
\end{enumerate}
\textbf{Show that Alice and Bob output the same key. Show a concrete attack on the security of the scheme by an eavesdropping adversary.}

\iffalse
\item {\bf (10 points) Public Key Encryption:} Consider the following public-key encryption scheme. The public key is $(G, q, g, h)$, where $h=g^x$ and the secret key is $x$, generated exactly as in the Elgamal encryption scheme. In order to encrypt a bit $b$, the sender does the following:
\begin{enumerate}
    \item If $b=0$, then choose a random $y\sample \mathbb{Z}_q$ and compute $c_1 =g^y$ and $c_2 =h^y$. The ciphertext is $(c_1,c_2)$.
    \item If $b = 1$, then choose independent random $y,z \sample \mathbb{Z}_q$, compute $c_1 = g^y$ and $c_2 = g^z$, and set the
    ciphertext equal to $(c_1, c_2)$.
\end{enumerate}
\textbf{Show that it is possible to decrypt efficiently given knowledge of $x$. Prove that this encryption
scheme is CPA-secure if the Decisional Diffie-Hellman problem is hard relative to $G$.}
\fi

\item {\bf (15 points) Digital Signatures:}
Let $(\mathsf{Gen,Sign,Verify})$ be a UF-CMA secure digital signature scheme for signing $n$-bit messages, and let $H:\{0,1\}^*\to\{0,1\}^n$ be a collision-resistant hash function. Consider the following new signature scheme $(\mathsf{Gen',Sign',Verify'})$ for signing arbitrary-length messages:
\begin{itemize}
    \item $\mathsf{Gen'}(1^{n})$: $(sk,vk)\leftarrow\mathsf{Gen}(1^{\lambda})$ and output $(sk,vk)$.
    \item $\mathsf{Sign'}(sk,m)$: Compute $\sigma \leftarrow \mathsf{Sign}(sk,H(m))$. Output $\sigma$.
    \item $\mathsf{Ver'}(vk,m,\sigma)$: Output 1 if $\mathsf{Ver}(vk,H(m),\sigma)$ outputs 1, else output 0.
\end{itemize}
\textbf{ Prove that $(\mathsf{Gen',Sign',Verify'})$ is also a UF-CMA secure digital signature scheme.}

\item {\bf (15 points) Oblivious Transfer and Public Key Encryption:} A two-message oblivious transfer between a receiver $R$ and a sender $S$ is defined by a tuple of 3 PPT algorithms $(\ot_R,\ot_S,\ot_\out)$.  The OT protocol works as follows (let $\secParam$ be the security parameter):
\begin{enumerate}
    \item \textbf{Receiver:} The receiver computes $(\msg_R,\rho)\gets\ot_R(1^\secParam, b)$, where $b\in\{0,1\}$ is the receiver’s input. It sends $\msg_R$ to the sender.
    \item \textbf{Sender:} The sender computes $\msg_S\gets \ot_S(1^\secParam,\msg_R,(m_0, m_1))$, where $m_0,m_1 \in \{0, 1\}^*$ are the sender’s input. The sender sends $\msg_S$ to the receiver.
    \item \textbf{Receiver's Output:} The receiver computes $m_b\gets\ot_\out(\rho,\msg_S)$. 
\end{enumerate}
This protocol satisfies the following properties:
\begin{itemize}
    \item \textbf{Correctness}:
        For each $m_0,m_1\in\{0,1\}^*$, $b\in\{0,1\}$, it holds that
        \[
            \Pr\left[\begin{split}
     (\rho,\msg_R)\gets\ot_R\left(1^\secParam,b\right)\\\msg_S\gets OT_S\left(1^\secParam,\msg_R,(m_0, m_1)\right)\end{split}\Biggm| \ot_\out\left(\rho,\msg_R,\msg_S\right) = m_b\right] = 1,
        \]

    \item \textbf{Security against Semi-Honest Sender}: It holds that, \begin{align*}
            \left\{ (\msg^0_R,\rho^0) \gets \ot_R\left(1^\secParam,0\right)\Bigm|\msg^0_R  \right\} \approx_c \left\{ (\msg^1_R,\rho^1) \gets \ot_R\left(1^\secParam,1\right)\mid\msg^1_R  \right\}
        \end{align*}
    \item \textbf{Security against Semi-Honest Receiver}: It holds that for each  $b\in\{0,1\}$, $m_0,m_1,m_0',m_1'\in\{0,1\}^*$, and $m_b=m'_{b}$,
     \begin{align*}
            \left\{\ot_S\left(1^\secParam,\msg_R,(m_0, m_1)\right) \right\} {\approx}_c  \left\{ \ot_S\left(1^\secParam,\msg_R,(m'_0, m'_1)\right)\right\}
        \end{align*}
        where $(\msg_R,\rho) \gets OT_R(1^\secParam,b)$.
\end{itemize}

\textbf{Public-Key Encryption Scheme:} Now consider the following construction of a public-key encryption scheme:
\begin{itemize}
    \item $\mathsf{KeyGen}(1^\lambda)$: Compute $(\msg_R,\rho)\gets\ot_R(1^\secParam, b)$. Set $pk=\msg_R$ and $sk=\rho$. Output $(pk,sk)$.
    \item $\mathsf{Enc}(pk,m)$: Compute $\msg_S\gets \ot_S(1^\secParam,pk,(m, m))$ and set $c=\msg_S$. Output ciphertext $c$.
    \item $\mathsf{Dec}(sk,c)$: Compute and output $m\gets\ot_\out(sk,c)$.
\end{itemize}

\textbf{Prove that the above construction $(\mathsf{KeyGen},\Enc,\Dec)$ is an IND-CPA secure public-key encryption.
}

\item {\bf (10 points) Pseudo-random Functions} 
Consider the following notion of ``Restricted PRFs'' consisting of three algorithms:
	\begin{itemize}
		\item $\mathsf{Gen}(1^n)$: is PPT algorithm that samples a PRF key $K \stackrel{\$}{\leftarrow} \{0,1\}^n$.
		\item $\mathsf{Restrict}(K,x)$: is PPT algorithm that takes as input a PRF key $K$ and a point $x$ in the input space of the PRF and outputs a restricted key $K_x$.
		\item $\mathsf{Eval}(K_x,x')$: is a deterministic polynomial time algorithm that takes as input a restricted key $K_x$ and an input $x'$ and outputs an element $y$.
	\end{itemize}
	We require two properties:
	
	\begin{itemize}
		\item \emph{Restricted Correctness}: on any input point $x'\neq  x$, PRF evaluation using the restricted key $K_x$ yields the same result as PRF evaluation using the (unrestricted) key $K$.
		\item \emph{Restricted Pseudorandomness}: the output of the PRF on input $x$ looks pseudorandom to any PPT adversary, even if the restricted key $K_x$ is given to the adversary.
	\end{itemize}


\textbf{Construction of Restricted PRF:}

Recall the binary tree based construction of PRFs discussed in class. Let $G:\{0,\}^n\to\{0,1\}^{2n}$ be a length doubling pseudorandom generator. We can imagine $G(s)=G_0(s)\|G_1(s)$, where $G_0(s)$ simply outputs the first $n$-bits of $G(s)$ and $G_1(s)$ outputs the last $n$-bits of $G(s)$. Let $x=x_1\|x_2\|\ldots\|x_n$ be an $n$-bit input. As discussed in class, the PRF  $F_K(x)=G_{x_n}\left(G_{x_{n-1}}\left(\ldots\left(G_{x_1}(K)\right)\right)\right)$. 

The PRF $F_K(x)$ can also be viewed as a binary tree of size $2^n$. The root corresponds to the PRF key $K$. The nodes at the first level are $K_0=G_0(K)$ and $K_1=G_1(K)$. Similarly, the nodes at the second level are $K_{00}=G_0(K_0)$, $K_{01}=G_1(K_0)$, $K_{10}=G_0(K_1)$ and $K_{11}=G_1(K_1)$. Nodes at the remaining levels can be defined in a similar way. The leaf nodes correspond to the outputs of the PRF.  

A restricted PRF can also be designed based on the above tree as follows:  

\begin{itemize}
		\item $\mathsf{Gen}(1^n)$: Sample a random $K \stackrel{\$}{\leftarrow} \{0,1\}^n$.
		\item $\mathsf{Restrict}(K,x)$: Let $x=x_1\|x_2\|\ldots\|x_n$ and let $\mathsf{Sibling}(y)$ denote the sibling of node $y$ in the above binary tree. $K_x=\mathsf{Sibling}(K_{x_1})\|\mathsf{Sibling}(K_{x_1x_2})\|\ldots\|\mathsf{Sibling}(K_{x_1\ldots x_n})\|x$
		\item $\mathsf{Eval}(K_x,x')$: Parse $K_x=\mathsf{Sibling}(K_{x_1})\|\mathsf{Sibling}(K_{x_1x_2})\|\ldots\|\mathsf{Sibling}(K_{x_1\ldots x_n})\|x$. Let $x=x_1\|x_2\|\ldots\|x_n$ and  $x'=x'_1\|x'_2\|\ldots\|x'_n$. Find the largest $\ell$ such that $x_1\|x_2\|\ldots\|x_\ell=x'_1\|x'_2\|\ldots\|x'_\ell$. Output $G_{x'_n}\left(G_{x'_{n-1}}\left(\ldots\left(G_{x'_{\ell+2}}(\mathsf{Sibling}\left(K_{x_1\ldots x_{\ell+1})} \right)\right)\right)\right)$
		
	\end{itemize}
In the above discussion we showed how to define the syntax and security of a PRF that is restricted on a single input and also gave a construction for such a PRF. We can also consider PRFs that are further restricted on more than 1 inputs. The syntax of $\mathsf{Restrict}$ and $\mathsf{Eval}$ algorithms for such PRFs and their correctness and pseudorandomness properties can be defined in a similar way.
\begin{enumerate}
    \item Show how you can restrict the key in the above tree-based construction such that the PRF is restricted on all inputs with a fixed 3-bit prefix, i.e., on inputs of the form $x=a_1\|a_2\|a_3\|x_4\|\ldots\|x_n$, where $a_1\|a_2\|a_3$ is a fixed value. 
    \item Show how you can restrict the key in the above tree-based construction such that the PRF is restricted on all inputs $X\leq x\leq Y$, where $X<Y$ and $X,Y\in \{0,1\}^n$ are fixed values.
\end{enumerate}

\item {\bf (10 points) Hash Functions} We saw in the homework that an {\em order-preserving} encryption scheme would be hard to achieve. Here we will show something similar for \emph{hash functions}.
Suppose a function $H : \bin^{2n} \mapsto\bin^n$ has the following property. For each $x,y\in\bin^{2n}$, if $x\leq y$, then $H(x)\leq H(y)$. Show that $H$ is not collision resistant (describe how to efficiently find a collision in such a function).

\item  \textbf{(15 points) Zero-Knowledge} Consider the following protocol to prove that $x \in L$, for some language $L$. Let $R$ be a associated relation (viewed as a circuit) such that $R(x,w)=1$ if and only if $w$ is a witness for the fact that $x \in L$. Let $R(x, r, w)$ denote the circuit such that $R(x, w) = r$ if and only if $w$ is a valid witness.

The verifier samples a random string $r$ and constructs a garbled version of the circuit $R(x, r, \cdot)$, with r and the statement $x$ fixed. It sends the garbled circuit to the prover. The prover and the verifier then run a series of OTs that allow the prover to learn the wire labels corresponding to $w$. The prover then uses the labels to evaluate the circuit, and sends the output to the verifier.
Finally, the verifier accepts if it received $r$, and rejects otherwise.

The full description is given below. For simplicity, assume that all valid witnesses are of the same length $\ell$. Let $w[i]$ deonte the $i$-th bit of a witness $w$. 
	
	\begin{center}
		\fbox{%
			\procedure{Protocol}{%
				\underline{\textbf{\prover}(x,w)} \> \> \underline{\textbf{\verifier}(x)}\\
                \> \> r \sample \lbrace 0, 1 \rbrace^\lambda \\
                \> \> \left(\widetilde{R}, \lbrace k^i_0,k^i_1 \rbrace_{i \in [\ell]} \right) \leftarrow \textsf{Garble}\Big(R(x, r,\cdot)\Big) \\
                \> \sendmessageleft*{\widetilde{R}} \\
                \> \sendmessage*{<->}{top={{\>\text{For } i \in 0,\ldots,\ell:\\ \> ~~\text{Use OT so $P$ learns $k^i_{w[i]}$}}}, style={dashed}} \\
                \text{Set }y \leftarrow {\sf Eval}(\widetilde{R}, \lbrace k^i_{w[i]} \rbrace_{i\in\ell}) \\
                \> \sendmessageright*{y} \\
                \> \> \text{ if } y = r\text{ then }{\sf Accept} \\
                \> \> \text{else }{\sf Reject}
			}
		}
		
	\end{center}
	
	\begin{enumerate}
        \item Prove that the above protocol satisfies correctness and soundness, with soundness error negligible in $\lambda$.
        \item Explain why the above protocol is zero-knowledge for a \emph{semi-honest} verifier.
        \item Explain why the above protocol is \textbf{not} zero-knowledge for a \emph{malicious} verifier.
    \end{enumerate}

    For (b) and (c), it is sufficient to explain in words why the above protocol is or is not zero-knowledge.
    
\end{enumerate}



\end{document}
