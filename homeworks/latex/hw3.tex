\documentclass[11pt]{article}
\usepackage{url,amsmath,setspace,amssymb,fullpage,color,varwidth,mdframed,mathtools}
\usepackage{enumitem}
\usepackage{xcolor}
\usepackage{csquotes}

\newcommand{\heading}[5]{
   \renewcommand{\thepage}{#1-\arabic{page}}
   \noindent
   \begin{center}
   \framebox[\textwidth]{
     \begin{minipage}{0.9\textwidth} \onehalfspacing
       {\bf CS 601.442/642 -- Modern Cryptography} \hfill #2

       {\centering \Large #5

       }\medskip

       {\it #3 \hfill #4}
     \end{minipage}
   }
   \end{center}
}

\newcommand{\scribe}[4]{\heading{#1}{#2}{}{ #4}{ #3}}

%\setlength{\parindent}{0in}

\newcommand{\proof}{{\bf Proof. }} %% To begin a proof write \proof
\newcommand{\qed}{\mbox{}\hspace*{\fill}\nolinebreak\mbox{$\rule{0.6em}{0.6em}$}} %%to end your proof write $\qed$.
\newcommand{\ma}{{\mathcal A}}
\newtheorem{lemma}{Lemma}
\newtheorem{theorem}[lemma]{Theorem}
\newtheorem{definition}{Definition}
\newtheorem{proposition}{Proposition}
\newtheorem{remark}{Remark}
\newtheorem{assumption}{Assumption}
\bibliographystyle{plain}

\begin{document}
\scribe{1}{Instructor:
Abhishek Jain}{Homework 4}{Deadline: October 12; 2022, 1:30 PM EST}


\section{Hard Core Predicate}

\begin{enumerate}

\item (10 points) Consider the following definition of a \textbf{2-bit hard core function}, which says that given the output of a OWF on an input $x$, it should be hard for the adversary to guess the 2-bit output of this hard core function on $x$:
\begin{mdframed}
A function $h:\{0,1\}^*\rightarrow\{0,1\}^2$ is a 2-bit hard-core function for $f(\cdot)$, if $h$ is efficiently computable given $x$ and there exists a negligible function $\nu$ s.t. for every non-uniform PPT adversary $\mathcal{A}$ and $\forall n\in\mathbb{N}$:
$$
\Pr\Big[x\leftarrow\{0,1\}^n: \mathcal{A}(1^n,f(x))=h(x)\Big]\leq\frac{1}{4}+\nu(n).
$$
\end{mdframed}
Let $f:\{0,1\}^{2n}\to\{0,1\}^{2n}$ be a OWF. Then we know that $g(x,r)=(f(x),r)$, where $|x|=|r|$ is also a OWF. Explain using a counterexample that $h(x,r)=\langle x[0:n],r\rangle\|\langle x[n:2n],r\rangle$, where $x[0:n]$ (and resp. $x[n:2n]$) denote the first $n$ bits (and resp. last $n$ bits) of $x$, is \textbf{NOT} a 2-bit hard core function for $g$.
\end{enumerate}

\section{Pseudorandom Functions}
\begin{enumerate}
   \item (10 points) Let $\{f_k\}_k$ be a family of PRFs. Is $\{g_k\}_k$ also a family of PRFs, where \\$g_k(x)$ = $f_{k}(x)||f_{k}(\bar{x})$? Prove via reduction or give a counterexample.
   \item (10 points) Let $\{f_k\}_k$ be a family of PRFs. Is $\{g_k\}_k$ also a family of PRFs, where \\$g_k(x)$ = $f_{k}(0||x)||f_{k}(1||x)$? Prove via reduction or give a counterexample.
   \item (15 points) Let $\{f_k\}_{k\in\{0,1\}^n}$ be a family of PRFs, where $f_k:\{0,1\}^n\to\{0,1\}^n$. Let $g:\{0,1\}^n\to\{0,1\}^{2n}$ be a PRG. Show via reduction that $\{h_k\}_{k\in\{0,1\}^n}$, where $h_k(x)=g(f_k(x))$ is also a family of PRFs.
\end{enumerate}

\section{Discrete Log}

\begin{enumerate}
    \item (10 points) Let $(G, \cdot)$ be a cyclic group with generator $g$.
    Suppose you are given $X \in G$. 
    You are allowed to choose any $X' \neq X$ and learn the discrete log of $X'$ (with respect to base $g$).
    Show that you can use this ability to learn the discrete log of $X$. 

\end{enumerate}

\section{Diffie Hellman}
\begin{enumerate}
    \item (10 points) Explain what is wrong with the following argument:
    \begin{displayquote}
        \textit{In Diffie-Hellman key agreement, Alice sends $A = g^a$ and Bob sends $B = g^b$. Their shared key is $g^{ab}$. To break the scheme, the eavesdropper can simply compute $A \cdot B = (g^a)\cdot(g^b) = g^{ab}$}
    \end{displayquote}
    \item (15 points) Let $G$ be a cyclic group of prime order $p$ with a generator $g$. 
    Recall that Decisional Diffie Hellman (DDH) assumption states that for $a,b,r \xleftarrow{\$} \{0,\dotsc,p-1\}$, the following distributions are computationally indistinguishable:
    $$
    \{g,g^a,g^b,g^{a \cdot b}\}\approx_c\{g,g^a,g^b,g^r\}
    $$
    Prove that for $a_1,a_2,b,r_1,r_2\xleftarrow{\$} \{0,\dotsc,p-1\}$, the following two distributions are indistinguishable under the DDH assumption:
    \begin{align*}
        D_1 &= \{g,g^{a_1},g^{a_2},g^{a_1\cdot b},g^{a_2 \cdot b}\}\\
        D_2 &= \{g,g^{a_1},g^{a_2},g^{r_1},g^{r_2}\}
    \end{align*}
    
     
\end{enumerate}

\end{document}
