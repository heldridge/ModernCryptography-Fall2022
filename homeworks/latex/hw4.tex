\documentclass[11pt]{article}
\usepackage{url,amsmath,setspace,amssymb,fullpage,color,varwidth,mdframed,mathtools}
\usepackage{enumitem}
\usepackage[n,advantage,operators,sets,adversary,landau,probability,notions,logic,ff,mm,primitives,events,complexity,asymptotics,keys]{cryptocode}

\newcommand{\heading}[5]{
   \renewcommand{\thepage}{#1-\arabic{page}}
   \noindent
   \begin{center}
   \framebox[\textwidth]{
     \begin{minipage}{0.9\textwidth} \onehalfspacing
       {\bf CS 601.442/642 -- Modern Cryptography} \hfill #2

       {\centering \Large #5

       }\medskip

       {\it #3 \hfill #4}
     \end{minipage}
   }
   \end{center}
}

\newcommand{\scribe}[4]{\heading{#1}{#2}{}{ #4}{ #3}}

%\setlength{\parindent}{0in}

\newcommand{\proof}{{\bf Proof. }} %% To begin a proof write \proof
\newcommand{\qed}{\mbox{}\hspace*{\fill}\nolinebreak\mbox{$\rule{0.6em}{0.6em}$}} %%to end your proof write $\qed$.
\newcommand{\ma}{{\mathcal A}}
\newtheorem{lemma}{Lemma}
\newtheorem{theorem}[lemma]{Theorem}
\newtheorem{definition}{Definition}
\newtheorem{proposition}{Proposition}
\newtheorem{remark}{Remark}
\newtheorem{assumption}{Assumption}
\bibliographystyle{plain}
\renewcommand{\sample}{\xleftarrow{\$}}
\newcommand{\Gen}{\mathsf{Gen}}
\newcommand{\Enc}{\mathsf{Enc}}
\newcommand{\Dec}{\mathsf{Dec}}
%\newcommand{\sk}{\mathsf{sk}}
%\newcommand{\pk}{\mathsf{pk}}
\newcommand{\Sign}{\mathsf{Sign}}
\newcommand{\Tag}{\mathsf{Tag}}
\newcommand{\Verify}{\mathsf{Verify}}
%\newcommand{\bin}{\{0,1\}}
\newcommand{\G}{{\mathbb G}}
\newcommand{\Z}{{\mathbb Z}}


\begin{document}
\scribe{1}{Instructor:
Abhishek Jain}{Homework 4}{Deadline: November 2; 2022, 1:30 PM EST}

\begin{enumerate}

\item In class we learned that single-message security does \textbf{not} imply multi-message security for secret-key encryption. Here we will prove that claim. 

Let $(\Gen, \Enc, \Dec)$ be a multi-message IND-CPA secure \textbf{secret-key} encryption scheme. Construct a secret-key encryption scheme $(\Gen', \Enc', \Dec')$ and prove that it is single-message IND-CPA secure but \textbf{not} multi-message IND-CPA secure.



\item
Let $\G$ be a cyclic group with order $q$ and generator $g$. Consider the following $\Gen$ and $\Enc$ functions for a public-key encryption scheme with single bit messages:
\begin{itemize}
    \item $\Gen(1^n):$ Sample $x \xleftarrow{\$} \Z_q$ and compute $h := g^x$. The public key is $h$ and the private key is $x$.
    \item $\Enc(h, m)$: 
    \begin{itemize}
        \item If $m == 0$ then sample $y \xleftarrow{\$} \Z_q$ and compute $c_1 := g^y$ and $c_2 := h^y$. The ciphertext is $(c_1, c_2)$.
        \item Else, if $m == 1$ then sample $y, z \xleftarrow{\$} \Z_q$ and compute $c_1 := g^y$ and $c_2 := g^z$. The ciphertext is $(c_1, c_2)$.
    \end{itemize}
\end{itemize}
\begin{enumerate}
    \item {\bf (10 points)} Write the decryption algorithm $\Dec(x, (c_1, c_2))$ and show that it is correct with overwhelming probability.
    \item {\bf (10 points)} Prove \textit{via reduction} that this encryption scheme is \textbf{IND-CPA secure} assuming that DDH is hard in $\G$.
\end{enumerate}

\item An {\em order-preserving} encryption scheme is a scheme where the ciphertexts follow the same lexicographic order as the messages. Such a property would be extremely useful for computing on encrypted databases. In this question, we will see why this property is hard to achieve. 
    
{\bf (10 points)} Let $\mathcal{E}\coloneqq(\Gen,\Enc,\Dec)$ be a public key encryption scheme such that for each $m_1,m_2\in\mathcal{M}$, if $m_1\leq m_2$, then $\Enc(\pk,m_1)\leq\Enc(\pk,m_2)$, where $\mathcal{M}$ is the message space and $\pk$ is the public key generated by the $\Gen$ algorithm. Show that $\mathcal{E}$ is \textbf{not} IND-CPA secure. 
    

\item {\bf (10 points)} Let $(\Gen,\Sign,\Verify)$ be a multi-message UF-CMA secure digital signature scheme that can be used to sign messages of length $n$. Consider the following new scheme for signing messages of length $2n$:
\begin{itemize}
    \item $\Gen'(1^n)$: Compute $(\sk_1,\pk_1)\gets\Gen(1^n)$ and $(\sk_2,\pk_2)\gets\Gen(1^n)$. Set $\sk\coloneqq(\sk_1,\sk_2)$ and $\pk\coloneqq(\pk_1,\pk_2)$. Output $(\sk,\pk)$.
    \item $\Sign'(m,\sk)$: Parse $\sk\coloneqq(\sk_1,\sk_2)$. Compute $\sigma_1\gets\Sign(m[0:n],\sk_1)$ and $\sigma_2\gets\Sign(m[n:2n],\sk_2)$. Output $\sigma\coloneqq\sigma_1||\sigma_2$.
    \item $\Verify'(\sigma,\pk)$: Parse $\pk\coloneqq(\pk_1,\pk_2)$ and $\sigma\coloneqq \sigma_1||\sigma_2$. Compute $b_1\gets\Verify(\sigma_1,\pk_1)$ and $b_2\gets\Verify(\sigma_2,\pk_2)$. Output $b\coloneqq b_1\wedge b_2$.
\end{itemize}
Show that $(\Gen',\Sign',\Verify')$ is \textbf{not} a UF-CMA secure digital signature scheme. 

\item 
\begin{enumerate}

    \item {\bf (10 points)} Let $(\Gen,\Sign,\Verify)$ be a multi-message UF-CMA secure digital signature scheme. Consider the following new scheme:
\begin{itemize}
    \item $\Gen'(1^n)$: Compute and output $(\sk,\pk)\gets\Gen(1^n)$.
    \item $\Sign'(m,\sk)$: Compute $\sigma\gets\Sign(m,\sk)$ and output $\sigma'\coloneqq\sigma||\sigma$.
    \item $\Verify'(\sigma,\pk)$: Parse $\sigma\coloneqq \sigma_1||\sigma_2$. Compute $b\gets\Verify(\sigma_1,\pk)$. If $\sigma_1=\sigma_2$ and $b=1$, output 1, else output 0.
\end{itemize}
Show that $(\Gen',\Sign',\Verify')$ is also a multi-message UF-CMA secure digital signature scheme. 
\item {\bf (10 points)} In the class we saw that PRFs imply MACs. You have to show that the converse is not true, i.e., a MAC scheme may not be a PRF. More specifically, given a UF-CMA secure MAC scheme $(\Gen,\Tag,\Verify)$, show that $(\Gen,\Tag)$ is not necessarily a PRF. 
\end{enumerate}

\end{enumerate}
\end{document}
