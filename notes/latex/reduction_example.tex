\documentclass[11pt]{article}
\usepackage{url,amsmath,setspace,amssymb,fullpage,color}
\usepackage{tikz}
\usepackage{pgfplots}
\usetikzlibrary{intersections}
\usetikzlibrary{matrix,shapes,arrows,positioning,chains, calc}
\usetikzlibrary{fit,shapes}

\usepackage[
n,
operators,
advantage,
sets,
adversary,
landau,
probability,
notions,	
logic,
ff,
mm,
primitives,
events,
complexity,
asymptotics,
keys]{cryptocode}

\usepackage{breqn}
\newcommand{\heading}[5]{
   \renewcommand{\thepage}{#1-\arabic{page}}
   \noindent
   \begin{center}
   \framebox[\textwidth]{
     \begin{minipage}{0.9\textwidth} \onehalfspacing
       {\bf CS 601.442/642 -- Modern Cryptography} \hfill #2

       {\centering \Large #5

       }\medskip

       {\it #3 \hfill #4}
     \end{minipage}
   }
   \end{center}
}

\newcommand{\scribe}[4]{\heading{#1}{#2}{}{ #4}{ #3}}

%\setlength{\parindent}{0in}
\newcommand{\zo}{\lbrace 0,1\rbrace}
\newcommand{\proof}{{\bf Proof. }} %% To begin a proof write \proof
\newcommand{\qed}{\mbox{}\hspace*{\fill}\nolinebreak\mbox{$\rule{0.6em}{0.6em}$}} %%to end your proof write $\qed$.
\newcommand{\ma}{{\mathcal A}}
\newtheorem{lemma}{Lemma}
\newtheorem{theorem}[lemma]{Theorem}
\newtheorem{definition}{Definition}
\newtheorem{proposition}{Proposition}
\newtheorem{remark}{Remark}
\newtheorem{assumption}{Assumption}
\bibliographystyle{plain}
\renewcommand{\sample}{\xleftarrow{\$}}
\begin{document}
\scribe{1}{Instructor:
Abhishek Jain}{Reduction Example}{}
\section{Pseudorandom Generators}
\subsection*{Problem}
Let $G:\{0,1\}^n\to\{0,1\}^{3n}$ be a PRG. Consider a function $H:\{0,1\}^n\to\{0,1\}^{6n}$ that works as follows:
$$
H(s): \text{First compute } s_1||s_2||s_3\coloneqq G(s) \text{, then compute and output } G(s_1)||G(s_3)  
$$
% $$
% H(s)= G(s_1)||G(s_2) \text{, where $s_1,s_2$ are computed as } s_1||s_2\coloneqq G(s) 
% $$
Prove via reduction that $H(\cdot)$ also a PRG.
\subsection*{Solution}
Consider the following hybrids:
\begin{itemize}
    \item $\mathcal{H}_0:$ $\{G(s_1)||G(s_3);s\sample\{0,1\}^n,s_1||s_2||s_3=G(s)\}$
    \item $\mathcal{H}_1:$ $\{G(s_1)||G(s_3);s_1,s_3\sample\{0,1\}^{n}\}$
    \item $\mathcal{H}_2:$ $\{G(s_1)||R_2;s_1\sample\{0,1\}^{n},R_2\sample\{0,1\}^{3n}\}$
    \item $\mathcal{H}_3:$ $\{R_1||R_2;R_1,R_2\sample\{0,1\}^{3n}\}$
\end{itemize}
In order to show that $H(s)$ is a PRG, it suffices to show that $\mathcal{H}_0$ is indistinguishable from $\mathcal{H}_3$. Let us assume for the for the sake of contradiction that $\mathcal{H}_0\not\approx\mathcal{H}_3$. In other words, let us assume that there exists an adversary $\mathcal{A}$ who can distinguish between $\mathcal{H}_0$ and $\mathcal{H}_3$ with some non-negligible advantage $\mu(n)$. From hybrid lemma, it follows that there must exist $i\in\{0,1,2\}$, such that $\mathcal{A}$ can distinguish between $\mathcal{H}_i$ and $\mathcal{H}_{i+1}$ with non-negligible advantage at least $\mu(n)/4$. We now show that if this is the case, then there exists another adversary $\mathcal{B}$ that can break the security of PRG $G$.

We argue this, by giving a proof via reduction for each $i\in\{0,1,2\}$. 

\begin{enumerate}
    \item Let $\mathcal{A}$ distinguish between $\mathcal{H}_0$ and $\mathcal{H}_{1}$ non-negligible advantage at least $\mu(n)/4$. We now construct another adversary $\mathcal{B}$ that breaks the security of $G$ as follows:

 \begin{bbrenv}{A}
		\begin{bbrbox}[name=$\bdv$, namepos=left]
			\pseudocode{
				y=G(s_1)\|G(s_3)\\
			}
			
			\begin{bbrenv}{B}
				
				\begin{bbrbox}[name=$\adv$,minheight=3.3cm,xshift=4cm,namepos=left]
					
				\end{bbrbox}
				
				\bbrmsgspace{.3cm}
				\bbrmsgto{top={$y$}}
				\bbrmsgspace{1cm}
				\bbrmsgfrom{top={$b'$}}
				\bbrmsgtxt{\pseudocode{
					\\
				}}
				
				
				
			\end{bbrenv}
			
		\end{bbrbox}
		
				
		\begin{bbrchallenger}{ChaA}[hdistance=3cm]
			\begin{bbrbox}[name=\underline{ Challenger for $G$ }, minwidth=3cm, namepos=center]
				\pseudocode{
				b\sample\{0,1\} \\
				\text{if }b=0\text{:}\\
	                \hspace{0.5 cm} s\sample\{0,1\}^n\\
					 \hspace{0.5 cm} s_1\|s_2\|s_3=G(s)\\
					 \text{else:}\\
					\hspace{0.5 cm} s_1\|s_2\|s_3\sample\{0,1\}^{3n}\\	\\
					\\
					\\
				}
			\end{bbrbox}
		\end{bbrchallenger}
		
		
		\bbrchallengerqryspace{1cm}
		\bbrchallengerqryto{ length=6cm, top={$s_1\|s_2\|s_3$}}		
		\bbrchallengerqryspace{3cm}
		\bbrchallengerqryfrom{top=$b'$}
	\end{bbrenv}
	
From the above reduction, it is clear that $\mathcal{B}$ has the same advantage in breaking $G$ as the advantage that $\mathcal{A}$ has in distinguishing between $\mathcal{H}_0$ and $\mathcal{H}_{1}$, which is at least $\mu(n)/4$. Since $\mu(n)/4$ is non-negligible, this would mean $\mathcal{B}$ can break $G$. However, we know that $G$ is a secure PRG, and hence, no such adversary can exist. Therefore our assumption was incorrect and $\mathcal{H}_0\approx\mathcal{H}_1$.
	%%%%%%%%%%%%%%%%%%%%%%%%%%%%%%%%%%%%%%%%%%%%%%%%%%%%%
	
\item Let $\mathcal{A}$ distinguish between $\mathcal{H}_1$ and $\mathcal{H}_{2}$ non-negligible advantage at least $\mu(n)/4$. We now construct another adversary $\mathcal{B}$ that breaks the security of $G$ as follows:

\begin{bbrenv}{A}
		\begin{bbrbox}[name=$\bdv$, namepos=left]
			\pseudocode{
			    s_1\sample\{0,1\}^n\\
				z=G(s_1)\|y\\
			}
			
			\begin{bbrenv}{B}
				
				\begin{bbrbox}[name=$\adv$,minheight=3.3cm,xshift=4cm,namepos=left]
					
				\end{bbrbox}
				
				\bbrmsgspace{.3cm}
				\bbrmsgto{top={$z$}}
				\bbrmsgspace{1cm}
				\bbrmsgfrom{top={$b'$}}
				\bbrmsgtxt{\pseudocode{
						\text{If } b'=0 \text{, set } b''=1 \\
						  \text{Else if } b'=1 \text{, set } b''=2 
				}}
				
				
				
			\end{bbrenv}
			
		\end{bbrbox}
		
				
		\begin{bbrchallenger}{ChaA}[hdistance=3cm]
			\begin{bbrbox}[name=\underline{ Challenger for $G$}, minwidth=3cm, namepos=center]
				\pseudocode{
				b\sample\{0,1\} \\
				\text{if }b=0\text{:}\\
	                \hspace{0.5 cm} s\sample\{0,1\}^n\\
					 \hspace{0.5 cm} y=G(s)\\
					 \text{else:}\\
					\hspace{0.5 cm} y\sample\{0,1\}^{3n}\\	\\
					\\
					\\
				}
			\end{bbrbox}
		\end{bbrchallenger}
		
		
		\bbrchallengerqryspace{1cm}
		\bbrchallengerqryto{ length=6cm, top={$y$}}		
		\bbrchallengerqryspace{3cm}
		\bbrchallengerqryfrom{top=$b''$}
	\end{bbrenv}
	%%%%%%%%%%%%%%%%%%%%%%%%%%%%%%%%%%%%%%%%%%%%%%%%%%%%%
	From the above reduction, it is clear that $\mathcal{B}$ has the same advantage in breaking $G$ as the advantage that $\mathcal{A}$ has in distinguishing between $\mathcal{H}_1$ and $\mathcal{H}_{2}$, which is at least $\mu(n)/4$. Since $\mu(n)/4$ is non-negligible, this would mean $\mathcal{B}$ can break $G$. However, we know that $G$ is a secure PRG, and hence, no such adversary can exist. Therefore our assumption was incorrect and $\mathcal{H}_1\approx\mathcal{H}_2$.
	
\item Let $\mathcal{A}$ distinguish between $\mathcal{H}_2$ and $\mathcal{H}_{3}$ non-negligible advantage at least $\mu(n)/4$. We now construct another adversary $\mathcal{B}$ that breaks the security of $G$ as follows:

\begin{bbrenv}{A}
		\begin{bbrbox}[name=$\bdv$, namepos=left]
			\pseudocode{
			    R_2\sample\{0,1\}^{3n}\\
				z=y\|R_2\\
			}
			
			\begin{bbrenv}{B}
				
				\begin{bbrbox}[name=$\adv$,minheight=3.3cm,xshift=4cm,namepos=left]
					
				\end{bbrbox}
				
				\bbrmsgspace{.3cm}
				\bbrmsgto{top={$z$}}
				\bbrmsgspace{1cm}
				\bbrmsgfrom{top={$b'$}}
				\bbrmsgtxt{\pseudocode{
						\text{If } b'=0 \text{, set } b''=2 \\
						  \text{Else if } b'=1 \text{, set } b''=3 
				}}
				
				
				
			\end{bbrenv}
			
		\end{bbrbox}
		
				
		\begin{bbrchallenger}{ChaA}[hdistance=3cm]
			\begin{bbrbox}[name=\underline{ Challenger for $G$}, minwidth=3cm, namepos=center]
				\pseudocode{
				b\sample\{0,1\} \\
				\text{if }b=0\text{:}\\
	                \hspace{0.5 cm} s\sample\{0,1\}^n\\
					 \hspace{0.5 cm} y=G(s)\\
					 \text{else:}\\
					\hspace{0.5 cm} y\sample\{0,1\}^{3n}\\	\\
					\\
					\\
				}
			\end{bbrbox}
		\end{bbrchallenger}
		
		
		\bbrchallengerqryspace{1cm}
		\bbrchallengerqryto{ length=6cm, top={$y$}}		
		\bbrchallengerqryspace{3cm}
		\bbrchallengerqryfrom{top=$b''$}
	\end{bbrenv}
	
From the above reduction, it is clear that $\mathcal{B}$ has the same advantage in breaking $G$ as the advantage that $\mathcal{A}$ has in distinguishing between $\mathcal{H}_2$ and $\mathcal{H}_{3}$, which is at least $\mu(n)/4$. Since $\mu(n)/4$ is non-negligible, this would mean $\mathcal{B}$ can break $G$. However, we know that $G$ is a secure PRG, and hence, no such adversary can exist. Therefore our assumption was incorrect and $\mathcal{H}_2\approx\mathcal{H}_3$.
\end{enumerate}

We have shown that $\mathcal{H}_0\approx\mathcal{H}_1\approx\mathcal{H}_2\approx\mathcal{H}_3$. Hence, our assumption must be wrong and there does not exist any adversary $\mathcal{A}$ who can distinguish between $\mathcal{H}_2$ and $\mathcal{H}_{3}$ with non-negligible advantage $\mu(n)$. Hence $H(\cdot)$ is a secure PRG.

\end{document}

